%------------------------------------------------------------------------------%
% bachelor thesis															   %
% create by: Mario Preishuber												   %
% create date: 2014, Jan 01.												   %
%------------------------------------------------------------------------------%
\begin{abstract}

\begin{table}[h]
\centering
\begin{tabular}{c}
\parbox{0.7\linewidth}{

Improving the performance of a \JS virtual machines is a hot topic. There are industry-standard benchmarks suits to support implementation and optimization of \JS virtual machines, but studies, like \cite{JSMeter2009}, illustrate that these benchmarks do not represent a real-world web application behavior. Since \JS is a fully garbage collected programming language improving performance of the memory management of a \JS virtual machine may improve performance in general. To be able to improve performance it requires better understanding performance deficiencies. To optimize a \JS virtual machine a configurable and realistic benchmarking of memory management is needed. Realistic benchmarking is only possible if typical \JS heaps are known. This
requires intensive analysis of \JS heaps of real-world applications.

We propose our analysis results of \JS heaps of real-world applications. Our
analysis results will support developer to implement realistic benchmarking suits which are able to simulate realistic \JS heap behavior. We observe 11 popular real-world web applications. Depending on the observed application we executed different user interactions. We use a sampling mechanism to generate frequently a snapshot of the current \JS heap. We analyze these snapshots about structure and distributions of object properties.

This thesis present an analysis of popular and \JS-intensive real-world web applications to obtain realistic distributions of object properties and heap structure properties.

}
\end{tabular}
\end{table}

\end{abstract}













	% 
	% We present our tool chain to obtain a realistic \JS heap model. We also 	present our analysis results of \JS heaps of real-world web applications. 	These analysis results can be used as base for implementing benchmarks which are able
	% to simulate a realistic \JS heap behavior. Furthermore, we illustrate the
	% overhead of Google's virtual machine, V8. We analyze the overhead of the V8
	% separate and present the distributions of system objects and distinguish by
	% types. At the end we compare system objects with mutator objects for better
	% illustration of the overhead.








