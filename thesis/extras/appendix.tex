\section{Appendix}

This section presents some additional tables about the generated heap snapshots.

\begin{table}[!htbp]
	\centering
	\begin{tabular}{|l||l|}
		\hline
		\textbf{Field name}	& \textbf{Description} 				\\ \hline \hline
		type & See Table \ref{tap:heap_snapshot_node_types}			\\ \hline
		name & A string which names the object. If the object is	\\
			 & an mutator variable this name is the same as in 		\\
			 & the mutator (reference to strings array)				\\ \hline	
		id	 & A unique id given by the virtual machine (V8)		\\ \hline
		selfsize  & This is the size of the object in byte. 		\\ \hline
		edgecount &	Number of reference which leaves the object.	\\ \hline	
	\end{tabular}
	\caption{Heap snapshot node fields}
	\label{tap:heap_snapshot_node_fields}
\end{table}
		
\begin{table}[!htbp]
	\centering
	\begin{tabular}{|l||l|}
		\hline
		\textbf{Field name}	&	\textbf{Description}			\\ \hline \hline
		type & See Table \ref{tap:heap_snapshot_edge_types}		\\ \hline
		name\_or\_index & A string which names the edge. 
							(reference to strings array)		\\	\hline
		to\_node		& It is the pointer to the child node. 
							(reference to nodes array)			\\	\hline 
	\end{tabular}
	\caption{Heap snapshot edge fields}
	\label{tap:heap_snapshot_edge_fields}
\end{table}
		
\begin{table}[!h]
	\centering
	\begin{tabular}{|l||l|}
		\hline
		\textbf{Name} & \textbf{Description} 					\\ \hline \hline
		context  & A variable from a function context			\\ \hline
		element  & An element of an array 						\\ \hline
		property & A named object property						\\ \hline
		internal &	A link that can't be accessed from JS, thus,\\
				 & its name isn't a real property name 
				  	(e.g. parts of a ConsString)				\\ \hline
		hidden	  & A link that is needed for proper sizes 
		 			calculation, 								\\
				  & but may be hidden from user					\\ \hline
		shortcut  & A link that must not be followed during	
					sizes calculation							\\ \hline
		weak 	  &	A weak reference (ignored by the GC)		\\ \hline
	\end{tabular}
	\caption{Heap snapshot edge types}
	\label{tap:heap_snapshot_edge_types}
\end{table}
		
\begin{table}[!htbp]
	\centering
	\begin{tabular}{|l||l|}
		\hline
		\textbf{Name} & \textbf{Description} 					\\ \hline \hline
		hidden & Hidden node, may be filtered when shown to user\\ \hline
		array  & An array of elements							\\ \hline
		string & A string										\\ \hline
		object & A JS object (except for arrays and strings)	\\ \hline
		code   & Compiled code									\\ \hline
		closure & Function closure								\\ \hline
		regexp & A regular expression							\\ \hline
		number & Number stored in the heap						\\ \hline
		native & Native object 									\\ \hline
		synthetic & Synthetic object, usually used for grouping 
					snapshot items together						\\ \hline
		concatenated string & Concatenated string 
							(a pair of pointers to strings)		\\ \hline
		sliced string & Sliced string 
						(a fragment of another string)			\\ \hline
	\end{tabular}
	\caption{Heap snapshot node types}
	\label{tap:heap_snapshot_node_types}
\end{table}

\begin{table}[!htbp]
	\centering
	\begin{tabular}{|l||l|}
		\hline
		\textbf{Name}	&	\textbf{Description}				\\ \hline \hline
		number of nodes & Each snapshot contains a field 
						  called \texttt{node\_count}.  			\\
						& The \texttt{node\_count} field is equal 
						  with \texttt{number of nodes} and			\\
						& shows how many nodes the heap currently 
						  contains. The \texttt{nodes\_count} 		\\
						& field is also an property of the table
						  \texttt{snapshots} and exists once		\\
						& per snapshot.								\\ \hline
		number of edges	& This is the same as \texttt{number of nodes} 
						  only for edges. Each snapshot 			\\
						& contains a field \texttt{edge\_count}. 
						  The \texttt{edge\_count} field is 		\\
						& equal with number of edges and shows how 
						  many edges the heap currently 			\\
						& contains. The \texttt{edge\_count} field 
						  is also an property of the table 			\\
						& \texttt{snapshots} and exists once per 
						  snapshot.									\\ \hline
		number of leafs	& The \texttt{number of leaves} is computed 
						  as followed: scan all nodes of a 			\\
						& snapshot and count those without children. 
						  Each node has a property called 			\\
						& \texttt{edge\_count} this value is 
						  different from the \texttt{edge\_count} 	\\
						& of a snapshot, only the name is the same. 
						  A node has no children if the 			\\
						& \texttt{edge\_count} is zero, because the
						  \texttt{edge\_count} field 				\\
						& contains the number of leaving edges. The
						  \texttt{edge\_count} field it 			\\
						& an property of the \texttt{nodes} table.	\\ \hline
		outdegree		& As explained before each node has a 
						  property \texttt{edge\_count} which 		\\
						& shows the number of leaving edges. 
						  The number of leaving edges is the same 	\\
						& as the outdegree of a node, so it follows 
						  that \texttt{edge\_count} is also 		\\
						& equal with the outdegree of a node. By 
						  using some aggregations of SQL it is 		\\
						& easy to compute the maximum, minimum and 
						  average number of leaving edges. 			\\
						& The \texttt{edge\_count} field it an 
						  property of the \texttt{nodes} table.		\\ \hline
		indegree		& The \texttt{indgeree} of a node is 
						  computed before inserting the data into	\\
						& the database. For each node in the 
						  snapshot is counted how many other nodes	\\
						& have a reference to this.					\\ \hline
		selfsize		& The \texttt{selfsize} is a property of a 
						  node in the snapshot and is 				\\
						& computed by the V8 snapshotting mechanism. 
						   It shows the size of a node in			\\
						& byte.										\\ \hline 
		rootdistance	& The \texttt{rootdistance} is an property 
						  of the \texttt{node} table and is 		\\
						& computed for each node before it is 
						  inserted in the database. 				\\
						& The rootdistance is the shortest path from 
						   any root to a node.						\\ \hline
		strongly		& To find strongly connected components in 
						  the heap graph the Tarjan's 				\\ 
		connected		& strongly connected components
			 			  algorithm\cite{Trajan} is used.			\\
		components		&											\\ \hline
	\end{tabular}
	\caption{Heap snapshot metrics}
	\label{tap:heap_snapshot_metrics}
\end{table}