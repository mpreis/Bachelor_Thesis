%------------------------------------------------------------------------------%
% bachelor thesis															   %
% create by: Mario Preishuber												   %
% create date: 2014, Jan 01.												   %
%------------------------------------------------------------------------------%

\section{Introduction}

\JS is part of many modern web applications and became an essential component
of modern web development. For \JS-intensive web applications it is
necessary that a browser provides a well performing \JS virtual machine. Since
\JS is a garbage collected programming language the memory management is a task
of the virtual machine. As a consequence, the performance of \JS virtual
machines may be enhanced by improving memory management performance. This
requires a better understanding of performance deficiencies.

Industry-standard benchmark suites support implementation and optimization of
\JS virtual machines, but studies, like \cite{JSMeter2009} illustrate that
these benchmarks do not represent a real-world web application behavior. To
optimize a \JS virtual machine, a configurable and realistic benchmarking of
memory management is needed. Realistic benchmarking is only possible if typical
\JS heaps are known.

We propose our analysis results of \JS heaps of real-world applications. First
we integrated a snapshotting functionality in Google's Chromium \cite{Chromium}
virtual machine V8 \cite{V8}. This mechanism periodically generates a snapshot
of the current \JS heap. We have studied 11 real-world web applications which
are similar to those in \cite{JSMeter2009}. Depending on the analyzed
application we performed different user interactions. The complete list of the
applications and user interactions is presented in Table
\ref{tab:real_world_apps}. Finally, we analyzed the structure and distributions
of object properties of these snapshots. Section \ref{sec:analysis_tools}
describes the tool chain we have used to generate and analyze snapshots in
detail.

The heap snapshots contain objects allocated by the \JS program (called mutator
in garbage collection terminology) and objects allocated by the virtual
machine. Section \ref{sec:analysis_mutator} describes our analysis of objects
allocated by a mutator, the so-called mutator heap and Section
\ref{sec:analysis_system}, takes a closer look at the objects allocated by the
virtual machine, the so-called system heap. Finally, Section
\ref{sec:analysis_sys_vs_mut} presents differences and similarities of mutator
and system heap. For both, mutator and virtual machine allocated objects, we
have analyzed the following properties and their distributions: object type,
size, lifetime, the number of outgoing edges, and the minimum distance from a
heap root to a object. This illustrates the allocation behavior of real-world
web applications and present the overhead of V8.

In this work we extend studies on the allocation behavior of real \JS web
applications \cite{JSMeter2009}. This thesis makes the following contribution:
An analysis of popular and \JS-intensive real-world web applications to obtain
realistic distributions of object properties and heap structure properties.

