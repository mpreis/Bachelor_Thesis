%------------------------------------------------------------------------------%
% bachelor thesis															   %
% create by: Mario Preishuber												   %
% create date: 2014, Jan 01.												   %
%------------------------------------------------------------------------------%

\section{Introduction}

\JS is part of many modern web applications and became to an essential
component of modern web development. Furthermore, \JS enables a dynamic,
individual behavior of a web application what is important for the user
experience. To realize such \JS intensive web applications it is necessary that
a browser provides a well performing \JS virtual machine. Since \JS is a fully
garbage collected programming language, like JAVA, the memory management is a
task of the virtual machine. As a consequence, the performance of \JS virtual
machines may be improved by improving their memory management performance. To
be able to improve performance it requires better understanding performance
deficiencies.

There are industry-standard benchmarks suits to support implementation and
optimization of \JS virtual machines, but studies, like \cite{JSMeter2009},
illustrate that these benchmarks do not represent a real-world web application
behavior. To optimize a \JS virtual machine a configurable and realistic
benchmarking of memory management is needed. Realistic benchmarking is only
possible if typical \JS heaps are known. This requires intensive analysis of
\JS heaps of real-world applications.

We propose our analysis results of \JS heaps of real-world applications. Our
analysis results can be used to implement a realistic benchmark which is able
to simulate a real-world \JS heap. We observe 11 real-world web applications
which are similar to those of \cite{JSMeter2009}. To extract information about
the \JS heap we used a sampling mechanism which frequent generates a snapshot
of the current \JS heap. This snapshotting functionality is integrated in
Google's Chromium \cite{Chromium} virtual machine V8 \cite{V8}. Depending on
the observed real-world application we perform different user interactions, the
complete list is presented by Table \ref{tab:real_world_apps}. During executing
such an user interaction snapshots a generated. Section
\ref{sec:analysis_tools} describes in detail the tool chain we used to generate
and analyze snapshots. We analyze these snapshots about structure and 
distributions of object properties. Our analysis is separated in three parts.

The first one, Section \ref{sec:analysis_mutator}, analyzes only the mutator
heap. A mutator is the common word of garbage collector people for any possible
application. In this section we only observe a subset of objects which are
allocated by the mutator and separated them by type.

The second part, Section \ref{sec:analysis_system}, takes a closer look at the
system heap. There are special object types which are allocated by the virtual
machine. All objects of these special type represent the system heap. These
objects are also part of a snapshot this allows us to use the same metics as in
Section \ref{sec:analysis_mutator}. We separate the objects again by type.

The third part, Section \ref{sec:analysis_sys_vs_mut}, compares the mutator and
the system heap. This analysis present differences and equalities of mutator
and system object distributions. In this section the objects are not separated
by there type instead we distinguish system and mutator objects. Our analysis
of this section illustrate the overhead which is produced by V8.

In this work we extend studies on the allocation behavior of real \JS web
applications \cite{JSMeter2009}. This thesis makes the following contribution:
An analysis of popular and \JS-intensive real-world web applications to obtain
realistic distributions of object properties and heap structure properties.
