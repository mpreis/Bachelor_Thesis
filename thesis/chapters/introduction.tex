%-------------------------------------------------------------------
% bachelor thesis
% create by: Mario Preishuber
% create date: 2014, Jan 01.
%-------------------------------------------------------------------

\section{Introduction}

\JS is essential in modern web development. The vendors of web browsers try to increase the performance of the implemented \JS virtual machines as good as possible. A reason for this intensive improvements is the huge impact of \JS features on the user experience during visiting a web application. The responsibility for a well performing \JS virtual machine bears by the browser vendors. This leaded to industry-standard benchmarking suites, which should help to optimize the \JS virtual machines. The problem is, studies, like \cite{JSMeter2009}, showed that this benchmarks do not represent a real-world web application behavior. If \JS virtual machines are optimized based on such benchmarks it might not improve the performance for realistic \JS heaps. To develop benchmarks which are able to simulate a behavior of a realistic web application it is necessary to know how a typical \JS heap looks like. 

We propose in this thesis our analysis results of \JS heaps of real-world web applications. In this work we present a way to analyze the heap of real-world web application as described in section \ref{sec:analysis_tools}. This section explains the used tools, the list of obtained web applications, the performed user interactions, and some first measurements results. Section \ref{sec:analysis_mutator} presents the analysis results of the obtained web application. We call this web applications mutator as for garbage collector people common. Furthermore, in our case a mutator can be any possible \JS application. The section \ref{sec:analysis_system} shows by using the same metics as in section \ref{sec:analysis_mutator} how the behavior of system object looks like. System objects are objects which are not allocated by the mutator. The virtual machine creates this object for different reasons. Note, we analyzed Google's V8, so the system objects only represent the unique behavior of V8. For instance there are so-called hidden classes which represent the properties of user-defined objects. At the end we compare the behavior of mutator objects and system objects in section \ref{sec:analysis_sys_vs_mut} to illustrate the impact of the virtual machine on the \JS heap.