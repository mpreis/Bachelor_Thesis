%-------------------------------------------------------------------
% bachelor thesis - presentation
%
% topic: ACDC4JS - How to analyze a JavaScript garbage collector
%
% create by: Mario Preishuber
% create date: 2014, Jan 01.
%-------------------------------------------------------------------
\section{Introduction}	
	\subsection{Memory management}
	\begin{frame}
		\frametitle{Memory management}
		\begin{columns}
			\begin{column}{.1\linewidth}
				\includegraphics<1>[width=5.5em]{../imgs/memory_0.pdf}
				\includegraphics<2>[width=5.5em]{../imgs/memory_1.pdf}
				\includegraphics<3>[width=5.5em]{../imgs/memory_2.pdf}
				\includegraphics<4>[width=5.5em]{../imgs/memory_3.pdf}
				\includegraphics<5>[width=5.5em]{../imgs/memory_4.pdf}
				\includegraphics<6>[width=5.5em]{../imgs/memory_5.pdf}
				\includegraphics<7>[width=5.5em]{../imgs/memory_6.pdf}
				\includegraphics<8>[width=5.5em]{../imgs/memory_6.pdf}
				\includegraphics<9>[width=5.5em]{../imgs/memory_6.pdf}
			\end{column}
			\begin{column}{.7\linewidth}
				\begin{itemize}
					\item Typical memory structure
					\begin{enumerate}
						\item Program code
						\item Constants (e.g. strings) 
						\item Heap
						\item Stack
					\end{enumerate}
				
					\pause
					\pause
					\pause
					\pause
					\pause
					\pause
					\pause
					
					\item Memory management in C
					\begin{itemize}
						\item \textbf{Allocation} explicit with \texttt{malloc()}
						\item \textbf{Deallocation} explicit with \texttt{free()}
					\end{itemize}
		
					\pause
		
					\item Memory management in JavaScript
					\begin{itemize}	
						\item \textbf{Allocation} explicit / implicit
						%\begin{itemize}
						%	\item explicit with \texttt{new}
						%	\item implicit with initialization 
						%\end{itemize}
						\item \textbf{Deallocation} implicit %with destroying references
					\end{itemize}
				\end{itemize}
			\end{column}
		\end{columns}
	\end{frame}
	
	\begin{frame}[fragile]
		\frametitle{Memory management}
		\begin{footnotesize}
			\begin{lstlisting}[language=JavaScript]
				// C
				char *s = malloc(4*sizeof(char)); // explicit allocation
				strncpy(s, "abc\0", 4);
				...
				access(s);
				...
				free(s);                          // explicit deallocation
			\end{lstlisting}			
			
			\pause
			
			\begin{lstlisting}[language=JavaScript]
				// JS
				function dosomething() {
				  var s = "abc";                  // implicit allocation
				  ...
				}                                 // destroy references 
				
				dosomething();
			\end{lstlisting} %(unreachable implicit deallocation)
		\end{footnotesize}		
	\end{frame}
	
	\begin{frame}
		\frametitle{Memory management}
		\begin{center}
			\includegraphics<1>[width=25em]{../imgs/app_gc_1.pdf}
			\includegraphics<2>[width=25em]{../imgs/app_gc_2.pdf}
			\includegraphics<3>[width=25em]{../imgs/app_gc_3.pdf}
			\includegraphics<4>[width=25em]{../imgs/app_gc_4.pdf}
			\includegraphics<5>[width=25em]{../imgs/app_gc_5_a.pdf}
			\includegraphics<6>[width=25em]{../imgs/app_gc_6_a.pdf}
			\includegraphics<7>[width=25em]{../imgs/app_gc_7.pdf}
			\includegraphics<8>[width=25em]{../imgs/app_gc_0.pdf}
		\end{center}
	\end{frame}
	
	\subsection{Garbage Collector} 
	\begin{frame}
		\frametitle{Garbage Collector}
		\begin{itemize}
			% live is an object since the allocation till the deallocation
			% dead is an object since the deallocation
			% unreachable is an object since there is NO MORE reference to it
			\item differs between live memory and dead memory
			\item deallocates dead memory for reuse
		%	\item computes unreachable memory
		%	\begin{itemize}
		%		\item Tracing
		%		\item Reference counting
		%	\end{itemize}
		%	\item free lists ...
		\end{itemize}
	\end{frame}
	
	\subsubsection{Difference between live and dead memory}
	\begin{frame}
		\frametitle{Difference between live and dead memory}
		\begin{center}
			\includegraphics<1>[width=30em]{../imgs/gc_mutator_objects_1.pdf}
			\includegraphics<2>[width=30em]{../imgs/gc_mutator_objects_2.pdf}
			\includegraphics<3>[width=30em]{../imgs/gc_mutator_objects_3.pdf}
			\includegraphics<4>[width=30em]{../imgs/gc_mutator_objects_4.pdf}
			\includegraphics<5>[width=30em]{../imgs/gc_mutator_objects_5.pdf}
		\end{center}
	\end{frame}
	
	\subsubsection{Tracing}
	\begin{frame}
		\frametitle{Tracing}
		\begin{center}
			\includegraphics<1>[width=20em]{../imgs/tracing_1.pdf}
%			\includegraphics<2>[width=20em]{../imgs/tracing_2.pdf}
%			\includegraphics<3>[width=20em]{../imgs/tracing_3.pdf}
%			\includegraphics<4>[width=20em]{../imgs/tracing_4.pdf}
%			\includegraphics<5>[width=20em]{../imgs/tracing_5.pdf}
%			\includegraphics<6>[width=20em]{../imgs/tracing_6.pdf}
			\includegraphics<2>[width=20em]{../imgs/tracing_7.pdf}
			\includegraphics<3>[width=20em]{../imgs/tracing_8.pdf}
			\includegraphics<4>[width=20em]{../imgs/tracing_9.pdf}
			\includegraphics<5>[width=20em]{../imgs/tracing_10.pdf}
			\includegraphics<6>[width=20em]{../imgs/tracing_11.pdf}
			\includegraphics<7>[width=20em]{../imgs/tracing_12.pdf}
			\includegraphics<8>[width=20em]{../imgs/tracing_13.pdf}
		\end{center}
	\end{frame}
	
	\subsubsection{Reference counting}
	\begin{frame}
		\frametitle{Reference counting}
		\begin{center}
			\includegraphics<1>[width=20em]{../imgs/ref_count_1.pdf}
			\includegraphics<2>[width=20em]{../imgs/ref_count_2.pdf}
			\includegraphics<3>[width=20em]{../imgs/ref_count_3.pdf}
			\includegraphics<4>[width=20em]{../imgs/ref_count_4.pdf}
		\end{center}
	\end{frame}
	